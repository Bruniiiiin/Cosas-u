\documentclass{article}
\usepackage{graphicx}
\usepackage{tikz}
\usepackage{pgfplots}
\usepackage{circuitikz}
\usepackage[spanish]{babel}
\usepackage{fancyhdr} 
\usepackage[bottom, norule]{footmisc}
\usepackage{multicol}    
\usepackage{sectsty}    
\usepackage{multicol}
\usepackage{amsmath}
\usepackage{hyperref}
\usepackage[square,numbers,sort&compress]{natbib}
\bibliographystyle{apsrev4-2}

\title{Informe Laboratorio 03}
\date{Septiembre 2025}
\pgfplotsset{compat=1.18}
\pgfplotsset{width=6cm}
\begin{document}
\fancyfoot[C]{}          
\fancyhead{}    
\begin{tikzpicture}[remember picture, overlay]

    \node[anchor=north west, xshift=2.6cm, yshift=-0.5cm] at (current page.north west) 
    {\includegraphics[width=3.5cm]{img/marcaderecha.png}};  
    \node[anchor=north east, xshift=-2.6cm, yshift=-1cm] at (current page.north east) 
    {\includegraphics[width=4cm]{img/logo-cfm.png}}; 

\end{tikzpicture}

\begin{center}
\large Laboratorio 1 \\
\vspace{0.3cm}
{\scshape\Huge Bobina Helmholtz y permeabilidad magnética \par}
\vspace{0.5cm}
{\scshape\Large Informe final \par}
\vspace{1.5cm}
{\large Nombres: Bruno Bustos, Darío Ferrada, Camilo Jordán, Catalina Zamorano}
\vspace{0.3cm}
\large 
\\ 
\ttfamily{Universidad de Concepción, Facultad de Ciencias Físicas y Matemáticas.}
\end{center}
\vspace{1cm}

\begin{abstract}

    En este informe se pretende utilizar una bobina de Helmholtz para medir la permeabilidad magnética del aire. 
    Se realizará un análisis teórico del campo magnético generado por la bobina, y se compararán los resultados experimentales 
    con los valores teóricos esperados analizando las posibles fuentes de error y su impacto en las mediciones. 

\end{abstract}

\begin{multicols}{2}

\section*{Introducción}

    Generalmente en problemas de electromagnetismo se utilizan variadas formulas para calcular todo tipo de campos magnéticos,
    sin embargo es necesario conocer que es lo que contienen estas formulas, en especifico nos interesaremos en la constante
    de permeabilidad magnética de algun material, la cual es una constante física que describe la capacidad del dicho material
    para permitir el paso de un campo magnético. Esta constante es fundamental en la teoría del electromagnetismo y aparece en varias
    ecuaciones importantes, como la ley de Ampère y la ley de Faraday. En este informe se pretende medir la permeabilidad magnética del
    aire utilizando una bobina de Helmholtz y compararla con el valor teórico conocido.

\newpage

\section*{Modelo Teórico}

    Partiremos con la ecuacion que describe el campo magnético generado por una espira circular de radio $R$ y corriente $I$ en su
    centro \cite{Griffiths}:

        \begin{equation}
            B(x) = \frac{\mu I R^2}{2(R^2 + x^2)^{3/2}}
        \end{equation}

    Consideraremos el punto medio entre las dos bobinas como el origen del sistema de coordenadas, por lo que cada bobina estará
    ubicada a una distancia $x_1 = R/2$ y $x_2 = -R/2$. Por lo tanto, el campo magnético en el centro de la bobina de Helmholtz
    será la suma de los campos magnéticos generados por cada bobina evaluados en el origen:
        
        \begin{equation}
            B_{total} = B_1(0) + B_2(0)
        \end{equation}

    Como ambas son simétricas ambas producen el mismo campo magnético en el centro $B_{total} =2B(x=R/2)$, evaluando tenemos que:

        \begin{align}
            B(\frac{R}{2})  &= \frac{\mu I R^2}{2(R^2 + (\frac{R}{2})^2)^{3/2}} \\
                            &= \frac{\mu I R^2}{2(R^2 + \frac{R^2}{4})^{3/2}} \\
                            &= \frac{\mu I R^2}{2(\frac{5R^2}{4})^{3/2}} \\
                            &= \frac{1}{2} \frac{\mu I}{R} \left(\frac{4}{5}\right)^{3/2}
        \end{align}

    Recordando que tenemos dos bobinas, el campo magnético total en el centro de la bobina de Helmholtz es:

        \begin{equation}
            B_{total} = 2B(\frac{R}{2}) = \mu \frac{NI}{R} \left(\frac{4}{5}\right)^{3/2}
        \end{equation}

    Llegando a la ecuacion del campo magnético generado por una bobina de Helmholtz \cite{bobinaHz} cuando se encuentra en su centro:

    \begin{equation}
        B = \mu_a \frac{NI}{R} \left(\frac{4}{5}\right)^{3/2}
    \end{equation}

    Donde $B$ es el campo magnético en el centro de la bobina, $\mu_a$ es la permeabilidad magnética del aire, $N$ es el número de 
    vueltas, $I$ es la corriente que pasa por la bobina y $R$ es el radio de la bobina.

    Podemos encontrar una relación lineal despejando $\mu_a$:

        \begin{equation}
            \mu_a = \frac{R}{NI} \left(\frac{4}{5}\right)^{-3/2} B
        \end{equation}

    Podemos decir $\frac{R}{NI} \left(\frac{4}{5}\right)^{-3/2} = H$, por lo que la ecuación queda:

        \begin{equation}
            \mu_a = H B
        \end{equation}

    %recordar : que H*E = m*c**2, donde H es el amor
    
\section*{Materiales}

    \begin{itemize}
        \item Bobina de Helmholtz
        \item Fuente de alimentación DC
        \item Dos multímetros
        \item Sensor de campo magnético
        \item Cables de conexión
    \end{itemize}

\section*{Montaje}



\section*{Análisis}



\section*{Conclusión}


\bibliography{referencias}


\end{multicols}
\end{document}