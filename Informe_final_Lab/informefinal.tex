\documentclass{article}
\usepackage{graphicx}
\usepackage{tikz}
\usepackage{pgfplots}
\usepackage{circuitikz}
\usepackage[spanish]{babel}
\usepackage{fancyhdr} 
\usepackage[bottom, norule]{footmisc}
\usepackage{multicol}    
\usepackage{sectsty}    
\usepackage{multicol}
\usepackage{amsmath}
\usepackage{hyperref}
\usepackage[square,numbers,sort&compress]{natbib}
\usepackage[hypcap=false]{caption}
\bibliographystyle{apsrev4-2}

\title{Informe Laboratorio 03}
\date{Septiembre 2025}
\pgfplotsset{compat=1.18}
\pgfplotsset{width=6cm}
\begin{document}
\fancyfoot[C]{}          
\fancyhead{}    
\begin{tikzpicture}[remember picture, overlay]

    \node[anchor=north west, xshift=2.6cm, yshift=-0.5cm] at (current page.north west) 
    {\includegraphics[width=3.5cm]{img/marcaderecha.png}};  
    \node[anchor=north east, xshift=-2.6cm, yshift=-1cm] at (current page.north east) 
    {\includegraphics[width=4cm]{img/logo-cfm.png}}; 

\end{tikzpicture}

\begin{center}
\large Laboratorio I \\
\vspace{0.3cm}
{\scshape\Huge Bobina Helmholtz y permeabilidad magnética \par}
\vspace{0.5cm}
{\large Nombres: Bruno Bustos, Darío Ferrada, Camilo Jordán, Catalina Zamorano}
\vspace{0.3cm}
\large 
\\ 
\ttfamily{Universidad de Concepción, Facultad de Ciencias Físicas y Matemáticas.}
\end{center}
\vspace{1cm}

\begin{abstract}

    En este informe se pretende utilizar una bobina de Helmholtz para medir la permeabilidad magnética del aire. 
    Se realizará un análisis teórico del campo magnético generado por la bobina, y se compararán los resultados experimentales 
    con los valores teóricos esperados, analizando las posibles fuentes de error y su impacto en las mediciones. 

\end{abstract}

\begin{multicols}{2}

\section*{Introducción}

    Generalmente en problemas de electromagnetismo se utilizan variadas fórmulas para calcular todo tipo de campos magnéticos,
    sin embargo es necesario conocer qué es lo que contienen estas fórmulas, en específico nos interesaremos en la constante
    de permeabilidad magnética de algún material, la cual es una constante física que describe la capacidad de dicho material
    para permitir el paso de un campo magnético. Esta constante es fundamental en la teoría del electromagnetismo y aparece en varias
    ecuaciones importantes, como la ley de Ampère y la ley de Faraday. En este informe se pretende medir la permeabilidad magnética del
    aire utilizando una bobina de Helmholtz para luego comparar los resultados obtenidos con el valor teórico conocido.

\newpage

\section*{Modelo Teórico}

    Partiremos con la ecuación que describe el campo magnético generado por una espira circular de radio $R$ y corriente $I$ en su
    centro \cite{Griffiths}:

        \begin{equation}
            B(x) = \frac{\mu I R^2}{2(R^2 + x^2)^{3/2}}.
        \end{equation}

    Consideraremos el punto medio entre las dos bobinas como el origen del sistema de coordenadas, por lo que cada bobina estará
    ubicada a una distancia $x_1 = R/2$ y $x_2 = -R/2$. Por lo tanto, el campo magnético en el centro de la bobina de Helmholtz
    será la suma de los campos magnéticos generados por cada bobina, evaluados en el origen:
        
        \begin{equation}
            B_{\text{total}} = B_1(0) + B_2(0).
        \end{equation}

        Como ambas bobinas son simétricas, producen el mismo campo magnético en el centro, esto es $B_{\text{total}} =2B(x=R/2)$. Evaluando tenemos que:

        \begin{align}
            B(\frac{R}{2})  &= \frac{\mu I R^2}{2(R^2 + (\frac{R}{2})^2)^{3/2}}, \\
                            &= \frac{\mu I R^2}{2(R^2 + \frac{R^2}{4})^{3/2}}, \\
                            &= \frac{\mu I R^2}{2(\frac{5R^2}{4})^{3/2}}, \\
                            &= \frac{1}{2} \frac{\mu I}{R} \left(\frac{4}{5}\right)^{3/2}.
        \end{align}

    Recordando que tenemos dos bobinas, el campo magnético total en el centro de la bobina de Helmholtz es:

        \begin{equation}
            B_{\text{total}} = 2B(\frac{R}{2}) = \mu \frac{NI}{R} \left(\frac{4}{5}\right)^{3/2}.
        \end{equation}

    Llegando a la expresión del campo magnético generado por una bobina de Helmholtz \cite{bobinaHz} cuando se encuentra en su centro:

    \begin{equation}
        B = \mu_\text{a} \frac{NI}{R} \left(\frac{4}{5}\right)^{3/2}.
    \end{equation}

    Donde $B$ es el campo magnético en el centro de la bobina, $\mu_\text{a}$ es la permeabilidad magnética del aire, $N$ es el número de 
    vueltas, $I$ es la corriente que pasa por la bobina y $R$ es el radio de la bobina.

    Podemos encontrar una relación lineal entre la corriente $I$ y el campo magnético $B$:

        \begin{equation}\label{eq:alpha}
            B = \alpha I , \quad \alpha = \mu_a \frac{N}{R} \left(\frac{4}{5}\right)^{3/2}.
        \end{equation}

    Siendo $\alpha$ la pendiente de la ecuación de $B$. También se utilizará la ley de Ohm para relacionar la corriente $I$ con el voltaje $V$ aplicado a la bobina:

        \begin{equation}
            I = \frac{V}{R_{b}}.
        \end{equation}

    Además, la corriente $I$ en paralelo se suma de la siguiente manera:

        \begin{equation}
            I = I_1 + I_2 + ... + I_n.
        \end{equation}

%recordar : que H*E = m*c**2, donde H es el amor 
    
\section*{Materiales}

    \begin{itemize}
        \item Bobina de Helmholtz
        \item Fuente de alimentación DC
        \item Multímetro
        \item Cables de conexión
        \item Magnetómetro
    \end{itemize}

\section*{Montaje}

    Se conectan dos bobinas de Helmholtz que tienen $N=77$ vueltas cada una, además de una resistencia total de 
    $R_\text{t} = 0.8 \Omega$ ya que están en paralelo. Se conecta la fuente de alimentación DC a las bobinas y se utiliza un multímetro 
    para medir el voltaje aplicado. En el centro de las bobinas, a un radio de $R = 0.2 m$, se coloca un sensor de campo magnético 
    para medir el campo $B$ generado por las bobinas al variar el voltaje aplicado. Mediante el uso del magnetómetro se calibra el sensor de campo magnético para obtener mediciones precisas.


        \begin{center}
        \captionsetup{type=figure}
        \includegraphics[width=0.7\columnwidth]{img/bobina.jpg}
        \caption{Bobina en configuración de Helmholtz}
        \label{fig:bobina}
        \end{center}

\section*{Resultados}

    A continuación se presentan los datos obtenidos durante el experimento, donde se midió el campo magnético $B$ en función de la 
    corriente, utilizando la ley de Ohm para relacionar el voltaje aplicado y la corriente en la bobina. Notar que el voltaje, la 
    corriente y el campo magnético tienen errores instrumentales de:

    \begin{equation}
        V_{ei} = 5 \times 10^{-4}~\text{V},
    \end{equation}

    \begin{equation}
        I_{ei} = 5 \times 10^{-4}~\text{A},
    \end{equation}

    \begin{equation}
        B_{ei} = 1 \times 10^{-5}~\text{mT}.
    \end{equation}
    
\begin{center}
\small
\centering
\label{tab:datos}
\begin{tabular}{|c|c|c|c|}
\hline
\textbf{Vol V} & \textbf{I Total A} & \textbf{I Bobina A} & \textbf{B mT} \\
\hline
0.115 & 0.144 & 0.072 & 0.0200 \\ \hline
0.230 & 0.288 & 0.144 & 0.0450 \\ \hline
0.340 & 0.425 & 0.213 & 0.0615 \\ \hline
0.461 & 0.576 & 0.288 & 0.0750 \\ \hline
0.569 & 0.711 & 0.356 & 0.1000 \\ \hline
0.674 & 0.843 & 0.422 & 0.1150 \\ \hline
0.787 & 0.984 & 0.492 & 0.1250 \\ \hline
0.885 & 1.106 & 0.553 & 0.1500 \\ \hline
1.008 & 1.260 & 0.630 & 0.1600 \\ \hline
1.109 & 1.386 & 0.693 & 0.1800 \\ \hline
1.222 & 1.527 & 0.764 & 0.1950 \\ \hline
1.325 & 1.656 & 0.828 & 0.2150 \\ \hline
1.435 & 1.794 & 0.897 & 0.2250 \\ \hline
1.550 & 1.938 & 0.969 & 0.2500 \\ \hline
1.659 & 2.074 & 1.037 & 0.2650 \\ \hline
1.774 & 2.218 & 1.109 & 0.2850 \\ \hline
1.864 & 2.330 & 1.165 & 0.3000 \\ \hline
1.984 & 2.480 & 1.240 & 0.3150 \\ \hline
2.097 & 2.621 & 1.311 & 0.3350 \\ \hline
2.203 & 2.754 & 1.377 & 0.3500 \\ \hline
2.320 & 2.900 & 1.450 & 0.3650 \\ \hline
2.425 & 3.031 & 1.516 & 0.3850 \\ \hline
2.558 & 3.198 & 1.599 & 0.4000 \\ \hline
2.650 & 3.313 & 1.657 & 0.4150 \\ \hline
2.750 & 3.438 & 1.719 & 0.4300 \\ \hline
2.873 & 3.591 & 1.796 & 0.4500 \\ \hline
2.990 & 3.738 & 1.869 & 0.4700 \\ \hline
3.075 & 3.844 & 1.922 & 0.4850 \\ \hline
3.210 & 4.013 & 2.007 & 0.5000 \\ \hline
3.322 & 4.153 & 2.077 & 0.5250 \\ \hline
3.659 & 4.574 & 2.287 & 0.5750 \\ \hline
3.767 & 4.709 & 2.355 & 0.5850 \\ \hline
3.880 & 4.850 & 2.425 & 0.6000 \\ \hline
\end{tabular}

\captionof{table}{Voltaje, corriente y campo magnético. }
\end{center}


\section*{Análisis}

    A partir de los datos obtenidos se realizará un gráfico de $B$ vs $I$ para obtener la pendiente $\alpha$ y así calcular la 
    permeabilidad magnética del aire. 
    Se grafican además los residuos de este modelo respecto a los datos para ver si los errores presentes son sistemáticos de modo que se deba mejorar el modelo.
        \begin{center}
        \captionsetup{type=figure}
        \includegraphics[width=1\columnwidth]{img/cosawena.png}
        \caption{Gráfico de campo magnético versus corriente.}
        \label{fig:graficoBI}
        \end{center}
    
        \begin{center}
        \captionsetup{type=figure}
        \includegraphics[width=1\columnwidth]{img/residuoslol.png}
        \caption{Gráfico de campo magnético versus corriente.}
        \label{fig:graficoBI}
        \end{center}

    Utilizando \ref{eq:alpha} y con algunos despejes algebraicos, podemos obtener el valor de la permeabilidad magnética del aire:

    \begin{equation}
        \mu_0 \cdot \frac{77}{0.2} \cdot  \left( \frac{4}{5} \right)^{3/2} = 2.464 \times 10^{-4}~\text{T/A},
    \end{equation}

    \begin{equation}
        \mu_0 \cdot 2.750 \cdot 0.846 =  2.464 \times 10^{-4}~\text{T/A},
    \end{equation}

    \begin{equation}
        \mu_0 = 8.945 \times 10^{-7} \text{H/m}.
    \end{equation}

    Como se puede observar en la figura \ref{fig:graficoBI}, los puntos se distribuyen de manera aleatoria. Además, analizando el error que nos entrega el modelo elegido, nos queda que la desviación estándar del coeficiente de posición y de la pendiente son:

    \begin{equation*}
        \sigma_{\text{pos}} =0.002 ~\text{mT/A},
    \end{equation*}

    \begin{equation*}
        \sigma_{\text{pend}} =0.001 ~\text{mT/A}.
    \end{equation*}

    Notar que el modelo no tiene coeficiente de posición, así que su desviación estándar puede indicar a perturbaciones de campos magnéticos externos al sistema propuesto.

    Ademas al comparar el valor obtenido con el valor teórico de la permeabilidad magnética del aire \cite{Griffiths} , que es $\mu_0 = 4\pi \times 10^{-7} \text{H/m}$,
    se puede observar que existe una discrepancia entre ambos valores. Esta discrepancia puede deberse a varios factores, como errores experimentales,
    imprecisiones en la medición del campo magnético, o incluso a la presencia de otros materiales en el entorno que puedan afectar las mediciones.
    El error relativo entre el valor teórico y el valor experimental se calcula de la siguiente manera:

    \begin{align}
    \varepsilon_r &= \left| \frac{\mu_{0} - \mu_{exp}}{\mu_{exp}} \right| \times \% 100 ,\\
    &= \left| \frac{3.61 \times 10^{-7}}{1.257 \times 10^{-6}} \right| \times \% 100 ,\\
    &= 28.7 \%.
    \end{align}

\section*{Predicción}

Para $I=7.0$ A, considerando los errores insrumentales:

\[
I = 7.0~\text{A},
\]

\[
\alpha = 2.464 \times 10^{-4}~\text{T/A},
\]

\[
\sigma_\alpha = 1.0 \times 10^{-7}~\text{T/A},
\]

\[
\sigma_I = 5 \times 10^{-4}~\text{A},
\]

\[
\sigma_{\text{pos}} = 2.0 \times 10^{-6}~\text{T},
\]

\[
\sigma_{B,\text{ei}} = 1.0 \times 10^{-8}~\text{T}.
\]


\begin{align}
    B   &= 2.464 \times 10^{-4} \cdot 7.0 ,  \\
        &= 1.7248 \times 10^{-3}\ \text{T}, \\
        &= 1.725\ \text{mT}.
\end{align}

Siendo este el campo magnético con una corriente de $7.0$ A.

\

Ahora veremos el error del campo magnético calculado. La incertidumbre se propaga como:
\[
\sigma_B = \sqrt{(I\sigma_\alpha)^2 + (\alpha\sigma_I)^2 + \sigma_{\text{pos}}^2 + \sigma_{B,\text{ei}}^2}.
\]

Sustituyendo valores:
\[
\sigma_B \approx 0.0021\ \text{mT}.
\]

Por lo tanto, el campo magnético con su error asociado es:
\[
B(7.0\ \text{A}) = 1.725 \pm 0.002\ \text{mT}.
\]

    Para determinar la incertidumbre experimental de la permeabilidad magnética ($\sigma_{\mu}$), se utiliza la propagación de errores considerando la incertidumbre de la pendiente obtenida del ajuste lineal ($\sigma_{\alpha}$) y la incertidumbre en la medición del radio ($\sigma_R \approx 5$ mm).

\begin{equation}
    \sigma_{\mu} = \mu_{exp} \sqrt{ \left( \frac{\sigma_{\alpha}}{\alpha} \right)^2 + \left( \frac{\sigma_R}{R} \right)^2 }
\end{equation}

Usando los valores del ajuste ($\alpha = 2.464 \times 10^{-4}$ T/A, $\sigma_{\alpha} = 1 \times 10^{-7}$ T/A) y la geometría ($R=0.2$ m):

\begin{align*}
    \sigma_{\mu} &\approx (8.945 \times 10^{-7}) \cdot 0.00643 \\
    \sigma_{\mu} &\approx 5.8 \times 10^{-9} \, \text{H/m}
\end{align*}

Por lo tanto, el resultado experimental se reporta como:

\begin{equation}
    \mu_{exp} = (8.95 \pm 0.06) \times 10^{-7} \, \text{H/m}
\end{equation}


\section*{Conclusiones}

    En este informe se ha logrado medir la permeabilidad magnética del aire utilizando una bobina de Helmholtz y compararla con el valor teórico conocido.
    A través del análisis de los datos experimentales, se obtuvo un valor de la permeabilidad magnética del aire que difiere del valor teórico en un 28.7\%.
    Esta discrepancia puede atribuirse a diversos factores, incluyendo errores experimentales y perturbaciones externas.
    A pesar de estas diferencias, el experimento proporciona una comprensión valiosa sobre el comportamiento del campo magnético generado por una bobina de Helmholtz
    y la importancia de la permeabilidad magnética en el electromagnetismo. 

\bibliography{referencias.bib}

\end{multicols}
\end{document}