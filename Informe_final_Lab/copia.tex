\documentclass{article}
\usepackage{graphicx}
\usepackage{tikz}
\usepackage{pgfplots}
\usepackage{circuitikz}
\usepackage[spanish]{babel}
\usepackage{fancyhdr} 
\usepackage[bottom, norule]{footmisc}
\usepackage{multicol}    
\usepackage{sectsty}    
\usepackage{multicol}
\usepackage{amsmath}
\usepackage{hyperref}
\usepackage[square,numbers,sort&compress]{natbib}
\usepackage[hypcap=false]{caption}
\bibliographystyle{apsrev4-2}

\title{Informe Laboratorio}
\date{Diciembre 2025}
\pgfplotsset{compat=1.18}
\pgfplotsset{width=6cm}
\begin{document}
\fancyfoot[C]{}          
\fancyhead{}    
\begin{tikzpicture}[remember picture, overlay]

    \node[anchor=north west, xshift=2.6cm, yshift=-0.5cm] at (current page.north west) 
    {\includegraphics[width=3.5cm]{img/marcaderecha.png}};  
    \node[anchor=north east, xshift=-2.6cm, yshift=-1cm] at (current page.north east) 
    {\includegraphics[width=4cm]{img/logo-cfm.png}}; 

\end{tikzpicture}

\begin{center}
\large Laboratorio I \\
\vspace{0.3cm}
{\scshape\Huge Bobina Helmholtz y permeabilidad magnética \par}
\vspace{0.5cm}
{\scshape\Large Informe final \par}
\vspace{1.5cm}
{\large Nombres: Bruno Bustos, Darío Ferrada, Camilo Jordán, Catalina Zamorano}
\vspace{0.3cm}
\large 
\\ 
\ttfamily{Universidad de Concepción, Facultad de Ciencias Físicas y Matemáticas.}
\end{center}
\vspace{1cm}

\begin{abstract}

    En este informe se pretende utilizar una bobina de Helmholtz para medir la permeabilidad magnética del aire. 
    Se realizará un análisis teórico del campo magnético generado por la bobina, y se compararán los resultados experimentales 
    con los valores teóricos esperados analizando las posibles fuentes de error y su impacto en las mediciones. 

\end{abstract}

\begin{multicols}{2}

\section*{Introducción}

    Generalmente en problemas de electromagnetismo se utilizan variadas formulas para calcular todo tipo de campos magnéticos,
    sin embargo es necesario conocer que es lo que contienen estas formulas, en especifico nos interesaremos en la constante
    de permeabilidad magnética de algun material, la cual es una constante física que describe la capacidad del dicho material
    para permitir el paso de un campo magnético. Esta constante es fundamental en la teoría del electromagnetismo y aparece en varias
    ecuaciones importantes, como la ley de Ampère y la ley de Faraday. En este informe se pretende medir la permeabilidad magnética del
    aire utilizando una bobina de Helmholtz y compararla con el valor teórico conocido.

\newpage

\section*{Modelo Teórico}

    Partiremos con la ecuacion que describe el campo magnético generado por una espira circular de radio $R$ y corriente $I$ en su centro \cite{Griffiths}:

        \begin{equation}
            B(x) = \frac{\mu I R^2}{2(R^2 + x^2)^{3/2}}
        \end{equation}

    Consideraremos el punto medio entre las dos bobinas como el origen del sistema de coordenadas, por lo que cada bobina estará
    ubicada a una distancia $x_1 = R/2$ y $x_2 = -R/2$. Por lo tanto, el campo magnético en el centro de la bobina de Helmholtz
    será la suma de los campos magnéticos generados por cada bobina evaluados en el origen:
        
        \begin{equation}
            B_{total} = B_1(0) + B_2(0)
        \end{equation}

    Como ambas son simétricas ambas producen el mismo campo magnético en el centro $B_{total} =2B(x=R/2)$, evaluando tenemos que:

        \begin{align}
            B(\frac{R}{2})  &= \frac{\mu I R^2}{2(R^2 + (\frac{R}{2})^2)^{3/2}} \\
                            &= \frac{\mu I R^2}{2(R^2 + \frac{R^2}{4})^{3/2}} \\
                            &= \frac{\mu I R^2}{2(\frac{5R^2}{4})^{3/2}} \\
                            &= \frac{1}{2} \frac{\mu I}{R} \left(\frac{4}{5}\right)^{3/2}
        \end{align}

    Recordando que tenemos dos bobinas, el campo magnético total en el centro de la bobina de Helmholtz es:

        \begin{equation}
            B_{total} = 2B(\frac{R}{2}) = \mu \frac{NI}{R} \left(\frac{4}{5}\right)^{3/2}
        \end{equation}

    Llegando a la ecuacion del campo magnético generado por una bobina de Helmholtz \cite{bobinaHz} cuando se encuentra en su centro:

    \begin{equation}
        B = \mu_a \frac{NI}{R} \left(\frac{4}{5}\right)^{3/2}
    \end{equation}

    Donde $B$ es el campo magnético en el centro de la bobina, $\mu_a$ es la permeabilidad magnética del aire, $N$ es el número de 
    vueltas, $I$ es la corriente que pasa por la bobina y $R$ es el radio de la bobina.

    Podemos encontrar una relación lineal entre la corriente $I$ y el campo magnético $B$:

        \begin{equation}
            B = \alpha I , \quad \alpha = \mu_a \frac{N}{R} \left(\frac{4}{5}\right)^{3/2}
        \end{equation}

    Tambien se utilizará la ley de Ohm para relacionar la corriente $I$ con el voltaje $V$ aplicado a la bobina:

        \begin{equation}
            I = \frac{V}{R_{b}}
        \end{equation}

    Ademas de que la corriente $I$ en paralelo se suma de la siguiente manera:

        \begin{equation}
            I = I_1 + I_2 + ... + I_n
        \end{equation}

%recordar : que H*E = m*c**2, donde H es el amor
    
\section*{Materiales}

    \begin{itemize}
        \item Bobina de Helmholtz
        \item Fuente de alimentación DC
        \item multímetro
        \item Sensor de campo magnético
        \item Cables de conexión
        \item Gaussómetro
        \item UNILAB Magnetic flux density unit
    \end{itemize}

\section*{Montaje}

    Primero se realiza un boceto del montaje experimental, el cual se muestra en la figura.

        \begin{center}
        \captionsetup{type=figure}
        \includegraphics[width=0.7\columnwidth]{img/tiks.png}
        \caption{Montaje experimental}
        \label{fig:montaje}
        \end{center}

    Se conectan dos bobinas de Helmholtz que tienen $N=77$ vueltas cada una, además de una resistencia total de 
    $R_t = 0.8 \Omega$ ya que están en paralelo. Se conecta la fuente de alimentación DC a las bobinas y se utiliza un multímetro 
    para medir el voltaje aplicado. En el centro de las bobinas a un radio de $R = 0.2 [m]$ se coloca un sensor de campo magnético 
    para medir el campo $B$ generado por las bobinas al variar el voltaje aplicado. Mediante el uso del UNILAB Magnetic flux density 
    unit se calibra el sensor de campo magnético para obtener mediciones precisas.

        \begin{center}
        \captionsetup{type=figure}
        \includegraphics[width=0.7\columnwidth]{img/bobina.jpg}
        \caption{Bobina en configuración de Helmholtz}
        \label{fig:bobina}
        \end{center}
        
        \begin{center}
        \captionsetup{type=figure}
        \includegraphics[width=0.7\columnwidth]{img/circuito.jpg}
        \caption{Circuito del montaje experimental}
        \label{fig:circuito}
        \end{center}

\section*{Resultados}

    A continuación se presentan los datos obtenidos durante el experimento, donde se midió el campo magnético $B$ función de la 
    corriente utilizando ley de Ohm para relacionar el voltaje aplicado y la corriente en la bobina.

    \begin{center}
\small
\begin{tabular}{ccc}
\hline
\textbf{V [V]} & \textbf{I [A]} & \textbf{B [mT]} \\
\hline
0.115 & 0.144 & 0.0200 \\
0.230 & 0.288 & 0.0450 \\
0.340 & 0.425 & 0.0615 \\
0.461 & 0.576 & 0.0750 \\
0.569 & 0.711 & 0.1000 \\
0.674 & 0.843 & 0.1150 \\
0.787 & 0.984 & 0.1250 \\
0.885 & 1.106 & 0.1500 \\
1.008 & 1.260 & 0.1600 \\
1.109 & 1.386 & 0.1800 \\
1.222 & 1.527 & 0.1950 \\
1.325 & 1.656 & 0.2150 \\
1.435 & 1.794 & 0.2250 \\
1.550 & 1.938 & 0.2500 \\
1.659 & 2.074 & 0.2650 \\
1.774 & 2.218 & 0.2850 \\
1.864 & 2.330 & 0.3000 \\
1.984 & 2.480 & 0.3150 \\
2.097 & 2.621 & 0.3350 \\
2.203 & 2.754 & 0.3500 \\
2.320 & 2.900 & 0.3650 \\
2.425 & 3.031 & 0.3850 \\
2.558 & 3.198 & 0.4000 \\
2.650 & 3.313 & 0.4150 \\
2.750 & 3.438 & 0.4300 \\
2.873 & 3.591 & 0.4500 \\
2.990 & 3.738 & 0.4700 \\
3.075 & 3.844 & 0.4850 \\
3.210 & 4.013 & 0.5000 \\
3.322 & 4.153 & 0.5250 \\
3.659 & 4.574 & 0.5750 \\
3.767 & 4.709 & 0.5850 \\
3.880 & 4.850 & 0.6000 \\
\hline
\end{tabular}

\captionof{table}{Voltaje, corriente y campo magnético.}
\end{center}


\section*{Análisis}

    A partir de los datos obtenidos se realizara un grafico de $B$ vs $I$ para obtener la pendiente $\alpha$ y así calcular la 
    permeabilidad magnética del aire. 
        \begin{center}
        \captionsetup{type=figure}
        \includegraphics[width=1\columnwidth]{img/BI.pdf}
        \caption{Gráfico de campo magnético $B$ vs corriente $I$.}
        \label{fig:graficoBI}
        \end{center}
    
    La pendiente, dado nuestro modelo lineal, será $\mu_0 \frac{N}{R} \frac{4}{5}^{3/4} = 0.193$. Todas estos valores son fijos, 
    con N = 77 siendo el número de vueltas y R = 20 cm siendo el radio de las bobinas.

    Considerando esos datos se puede trabajar algebráicamente:

    \begin{equation}
        \mu_0 \cdot \frac{77}{20} \cdot \frac{4}{5}^{3/4} = 0.193
    \end{equation}

    \begin{equation}
        \mu_0 \cdot \frac{77}{20} \cdot \frac{4}{5}^{3/4} = 0.193
    \end{equation}

    \begin{equation}
        \mu_0 \cdot 2.750 \cdot 0.846 = 0.193
    \end{equation}

    \begin{equation}
        \mu_0 = 0.083
    \end{equation}

    Además, al modelo elegido se observó cuál es su residuo con respecto a los datos tomados para ver si existía algún error sistemático o si existe un modelo mejor que pueda representar nuestros datos.
     \begin{center}
        \captionsetup{type=figure}
        \includegraphics[width=1\columnwidth]{img/residuo.png}
        \caption{Residuos del Modelo.}
        \label{fig:graficoRM}
        \end{center} 

    Como se puede observar en \ref{fig:graficoRM} los puntos se mueven de manera aleatoria, por lo que nuestro modelo grafica decentemente nuestros datos.

    Analizando además el error que nos entrega el modelo mismo elegido, nos queda que la desviación estándar de la pendiente y el coeficiente de posición son:

    \begin{equation*}
        \sigma_{pos} =0.002
    \end{equation*}

    \begin{equation*}
        \sigma_{pend} =0.001
    \end{equation*}


\section*{Predicción}
Dado nuestro modelo y considerando los errores que este tiene, podemos realizar una predicción respecto a un campo magnético mucho más grande de los que trabajamos.


\section*{Conclusión}

\bibliography{referencias.bib}

\end{multicols}
\end{document}