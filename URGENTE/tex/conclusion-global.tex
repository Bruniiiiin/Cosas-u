\documentclass[../portafolio.tex]{subfiles}

\begin{document}

\chapter*{Conclusiones}
\addcontentsline{toc}{chapter}{Conclusiones}
\markboth{Conclusiones}{Conclusiones}

\hfill \textbf{Fecha de presentación:} Miércoles 10 de diciembre de 2025

\medskip

\textbf{\color{red} ESTE CAPÍTULO DEBE SER LLENADO AL FINALIZAR LA ASIGNATURA y por supuesto antes de la fecha de entrega}.

%--------------------------------------------------------------------------------
% Inicie con un resumen breve de cuáles eran los objetivos del portafolio;

%--------------------------------------------------------------------------------
% [Resumen de los contenidos]
% - Un resumen MUY breve de cuáles son las evidencias de aprendizaje que incluyó en este portafolio. Algo como "En el capítulo 1, se derivó numéricamente la función coseno, usando un esquema de derivadas centradas, para estudiar el error absoluto con respecto a la derivada analítica de la misma función."
% - Incluya una breve reflexión de lo que aprendió en cada actividad, lo que faltó aprender, lo que no se entendió y lo que sí se entendió bien.
% - Haga lo anterior por cada evidencia de aprendizaje.


%--------------------------------------------------------------------------------
% [Autoevaluación del alumno/a]
% Realice una reflexión de cómo trabajó usted, qué cree haber hecho bien y mal en el curso, qué le gustaría  hacer a futuro (en la forma de estudiar y en cómo cree que aplicará los contenidos de este portafolio en el futuro), cómo han distribuido su trabajo a lo largo del trabajo en este portafolio.

% --------------------------------------------------------------------------------
% [Evaluación del curso]
% - Realice una comparación entre sus expectativas iniciales, tal como las describió en la sección de presentación, y lo que realmente aprendió y experimentó a lo largo del curso. ¿Se cumplieron sus expectativas sobre la asignatura? ¿En qué medida?
% - ¿Qué aspectos del curso o del portafolio superaron, igualaron o no alcanzaron sus expectativas iniciales?
% - Agregue sugerencias para futuras versiones del curso, para que estudiantes de generaciones venideras se beneficien de una aplicación mejorada de este instrumento de evaluación.
% - ¿Cuál es la evidencia de este portafolio que usted cree es mejor/más relevante/en la que aprendió mejor? ¿Qué diferencia a esa evidencia del resto incluido en este portafolio?
% - ¿Puede evaluar la utilidad de este portafolio?


\textcolor{red}{(Este es solo un ejemplo. Debe editarlo y ajustarlo
  con su propio trabajo. Esta sección es responsabilidad del autor de este portafolio)}
El objetivo de este portafolio fue reunir y reflexionar sobre diversas
actividades de Física Computacional II, aplicando herramientas
numéricas y computacionales al análisis de problemas físicos.

En los distintos capítulos trabajé con esquemas de derivación numérica
y análisis de errores, exploré el comportamiento de algoritmos y
comparé resultados con soluciones analíticas. Aprendí a reconocer la
importancia de equilibrar truncamiento y redondeo, así como la
necesidad de explicar los procedimientos y no limitarme a mostrar solo
fórmulas o código. En algunos casos me resultó desafiante interpretar
los gráficos con precisión, pero en general logré comprender cómo se
relacionan los métodos computacionales con la física real.

En cuanto a mi propio trabajo, considero que fui constante en la
elaboración de los capítulos y logré mejorar mi claridad en las
explicaciones. Sin embargo, reconozco que a veces dejé para último
momento la redacción, lo cual afectó la profundidad de algunas
reflexiones. A futuro, me gustaría organizar mejor mis tiempos y
practicar más la validación de resultados con distintas estrategias.

Comparando mis expectativas iniciales con lo que aprendí, puedo decir
que se cumplieron en buena medida: esperaba consolidar mis habilidades
computacionales y terminé entendiendo mucho mejor cómo usarlas para
estudiar fenómenos físicos. Lo que más destaco es la evidencia donde
analicé el error de un esquema centrado, pues me permitió ver con
claridad cómo interactúan los distintos tipos de errores. Finalmente,
considero que este portafolio fue útil como herramienta de
aprendizaje, porque me obligó a reflexionar sobre mi proceso y a
reconocer tanto logros como dificultades.

\end{document}
