\documentclass[../portafolio.tex]{subfiles}

% Solo agregue paquetes en el preámbulo de ../portafolio.tex

\begin{document}

\chapter*{Información personal y académica}
\addcontentsline{toc}{chapter}{Información personal y académica}
\markboth{Información personal y académica}{Información personal y académica}


%%%%%%%%%%%%%%%%%%%%%%%%%%%%%%%%%%%%%%%%%%%%%%%%%%%%%%%%%%%%%%%%%%%%%%
% Llene todos los campos, respetando tildes, mayúsculas y minúsculas.
\section*{Datos personales}

\begin{description}
\item[{Nombre completo}] Bruno Esteban Bustos Flores % nombres y apellidos completos.
\item[{Matrícula}] 2023438600 % matrícula udec
\item[{Fecha de Nacimiento}] Lunes \#6, Noviembre de 2004 % día de mes de año
\item[{Nacionalidad}] Chilena
\item[{E-Mail institucional}] \href{mailto:brbustos2023@udec.cl}{brbustos2023@udec.cl}
\end{description}


%%%%%%%%%%%%%%%%%%%%%%%%%%%%%%%%%%%%%%%%%%%%%%%%%%%%%%%%%%%%%%%%%%%%%%
\section*{Breve biografía académica}
% Redacte una breve biografía (5 a 7 líneas) que incluya los
% siguientes aspectos:
% - Su nombre completo y el año en el que ingresó a la Universidad de
% Concepción.
% - Mencione su carrera actual y en qué año académico se encuentra.
% - Describa brevemente su trayectoria educativa previa a la universidad
% (por ejemplo, dónde cursó la educación media y cualquier logro académico
% relevante).
% - Mencione sus metas académicas y profesionales al finalizar el
% pregrado. ¿Qué le gustaría lograr al terminar la carrera? ¿En qué
% áreas le gustaría especializarse o trabajar?
% - Si lo considera pertinente, puede mencionar cualquier actividad
% extracurricular que haya contribuido a su formación (cursos,
% proyectos, trabajos, etc.).

Mi nombre es Bruno Esteban Bustos Flores y ingresé a la Universidad de
Concepción en el año 2023. Actualmente, curso la carrera de Ciencias
Físicas y me encuentro en mi cuarto semestre académico. Estudie en el 
Colegio Salesiano de Concepción durante mi educación media. Mi meta a 
futuro actualmente es estudiar los efectos cuanticos sobre nanomateriales.
Me gustaría que al finalizar mi pregrado pueda contar con las conexiones
necesarias para poder trabajar en el área de mi preferencia. 

%%%%%%%%%%%%%%%%%%%%%%%%%%%%%%%%%%%%%%%%%%%%%%%%%%%%%%%%%%%%%%%%%%%%%%
\section*{Visión general e interés sobre la asignatura}
% En esta sección, reflexione y describa:
% - ¿Cuál es su percepción inicial sobre la asignatura de Física
% Computacional II? ¿Cómo se relaciona con su formación académica y sus
% intereses?
% - ¿Qué habilidades o conocimientos espera desarrollar en esta
% asignatura, específicamente en el uso de herramientas computacionales
% aplicadas a la física?
% - ¿De qué manera cree que lo aprendido en esta asignatura contribuirá a
% su desempeño en otros cursos o en su carrera profesional a futuro?
% - Si tiene alguna expectativa específica o tema de interés particular
% dentro de la asignatura, menciónelo aquí.

Creo que la asignatura de Física Computacional II es fundamental para mi
formación académica, ya que me permitirá adquirir habilidades prácticas
en el uso de herramientas, especialmente computacionales, aplicadas a la
física, incluido en las áreas en las que me gustaría especializarme, como
el poder simular sistemas físicos o manipular datos experimentales de forma
más eficiente. Todo esto tambien se aplica a los siguientes cursos que
tendré que cursar en mi carrera, ya que michos de ellos requieren el
conocimiento basico de programación y análisis de datos que me entrega 
esta asignatura.

%%%%%%%%%%%%%%%%%%%%%%%%%%%%%%%%%%%%%%%%%%%%%%%%%%%%%%%%%%%%%%%%%%%%%%
\section*{Resultados esperados de este portafolio}
% En esta sección, reflexione sobre los resultados que espera obtener al
% realizar este portafolio. Puede incluir lo siguiente:
% - ¿Qué habilidades y conocimientos espera haber consolidado al completar
% este portafolio?
% - ¿Cómo cree que el portafolio le ayudará a organizar, analizar y
% aplicar los conceptos aprendidos durante la asignatura?
% - ¿De qué manera considera que este portafolio puede servirle como
% referencia o herramienta para su futura formación académica o
% profesional?
% - Reflexione sobre cómo el proceso de autoevaluación y la inclusión de
% evidencias le permitirá comprender mejor su propio progreso.

Espero que al completar el portafolio logre consolidar mis habilidades en
el uso de herramientas computacionales dictadas en la asignatura, así como
tambien aprender a organizarme de forma más eficiente y aprender el metodo
que mejor se me acomode para trabajar. Al ser un portafolio modular, podre 
organizar mis trabajos de forma cronologica, lo que me ayudara a ver mi
progreso a lo largo del semestre, de esta forma podre lograre recordar
lo aprendido en la asignatura y aplicarlo en mi futuro académico solo con
una revisión rápida de mi portafolio. Todo esto me ayudara a comprender
mejor mi propio progreso y a identificar áreas en las que pueda mejorar
mediante el uso de la autoevaluación. 

\end{document}
