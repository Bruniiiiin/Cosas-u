\documentclass{article}
\usepackage{graphicx} % Required for inserting images
\usepackage{tikz}
\usepackage{circuitikz}
\usepackage{pgfplots}
\usepackage[spanish]{babel}
\usepackage{fancyhdr} 
\usepackage[bottom, norule]{footmisc}
\usepackage{multicol}    
\usepackage{sectsty}    
\usepackage{multicol}
\usepackage{amsmath}
\usepackage{hyperref}
\title{Laboratorio 1}
\date{Octubre 2025}
\pgfplotsset{width=6cm}
\begin{document}
\fancyfoot[C]{}          
\fancyhead{}    
\begin{tikzpicture}[remember picture, overlay]
    \node[anchor=north west, xshift=2.6cm, yshift=-0.5cm] at (current page.north west) 
    {\includegraphics[width=3.5cm]{img/marcaderecha.png}};
    \node[anchor=north east, xshift=-2.6cm, yshift=-1cm] at (current page.north east) 
    {\includegraphics[width=4cm]{img/logo-cfm.png}}; 
\end{tikzpicture}

\begin{center}
\large Laboratorio 1 \\
\vspace{0.3cm}
{\scshape\Huge Tarea 2 \par}
\vspace{0.5cm}
{\scshape\Large Constante elástica de un resorte y Aceleración de gravedad \par}
\vspace{1.5cm}
{\large Autor: Bruno Bustos}
\vspace{0.3cm}
\large 
\\ 
\ttfamily{Universidad de Concepción, Facultad de Ciencias Físicas y Matemáticas.}
\end{center}
\vspace{1cm}

\begin{abstract}

    El objetivo de este informe es determinar la constante elástica de un resorte y la aceleración
    de gravedad mediante la medición experimental de la elongación de un resorte al colgarle diferentes
    masas y dejando caer un carrito a una distancia conocida, peso conocido y midiendo el tiempo que
    tarda en caer, respectivamente.

\end{abstract}

\begin{multicols}{2}

\section{Introducción}

En esta Tarea se aborda la determinación experimental de
dos constantes físicas fundamentales: La aceleración de gravedad $g$
y la constante elástica de un resorte $k$.\\
La aceleración de gravedad es un parámetro fundamental que describe la
fuerza con la que la Tierra atrae a los objetos hacia su centro, mientras
que la constante elástica de un resorte caracteriza la rigidez de un resorte
y su relación  con la fuerza aplicada según la Ley de Hooke.\\
A través de experimentos prácticos, se buscará medir estas constantes de
manera directa, evaluando los posibles errores y comparando los resultados
experimentales con los valores teóricos.

\section{Modelo Teórico}

    \subsection{Constante elástica de un resorte}

        La constante elástica de un resorte $k$ se define como la relación entre la fuerza $F$ aplicada al
        resorte y la elongación $\Delta x$ que experimenta, de acuerdo con la Ley de Hooke \ref{1}:

            \begin{equation}
                F = k \Delta x
            \end{equation}

        Donde:
            \begin{itemize}
                \item $F$ es la fuerza aplicada (en Newtons, N).
                \item $k$ es la constante elástica del resorte (en Newtons por metro, N/m).
                \item $\Delta x$ es la elongación del resorte (en metros, m).
            \end{itemize}
    \subsection{Aceleración de gravedad}
        La aceleración de gravedad $g$ se puede determinar mediante la medición del tiempo $t$ que tarda 
        un objeto en caer, en nuestro caso un carrito con ruedas que desprecian el roce, variando angulos
        de inclinación $\theta$ y conociendo la distancia $d$ recorrida. La relación entre estas variables
        está dada por la siguiente ecuación\ref{1}:

            \begin{equation}
                d = \frac{1}{2} a t^2
            \end{equation}

        Donde:
            \begin{itemize}
                \item $d$ es la distancia recorrida (en metros, m).
                \item $a$ es la aceleración del objeto (en metros por segundo al cuadrado, m/s²).
                \item $t$ es el tiempo de caída (en segundos, s).
            \end{itemize}

        En el caso de un plano inclinado, la aceleración $a$ está relacionada con la aceleración de 
        gravedad $g$ y el ángulo de inclinación $\theta$ mediante la siguiente relación \ref{1}:

            \begin{equation}
                a = g \sin(\theta)
            \end{equation}
        
\section{Materiales}

    \subsection{Materiales para la determinación de la constante elástica de un resorte}

    \subsection{Materiales para la determinación de la aceleración de gravedad}

\section{Procedimiento}

\subsection{Determinación de la constante elástica de un resorte}

\subsection{Determinación de la aceleración de gravedad}

\section{Análisis}

\section{Conclusión}

\section{Referencias}
    \begin{enumerate}
    
        \item \label{1} Resnick, R., Halliday, D., \& Krane, K. (1988). \textit{Physics, Vol. 1} (4th ed.). John Wiley \& Sons. 
    
    \end{enumerate}
\end{multicols}
\end{document}