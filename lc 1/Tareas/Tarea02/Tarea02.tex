\documentclass{article}
\usepackage{graphicx} % Required for inserting images
\usepackage{tikz}
\usepackage{circuitikz}
\usepackage{pgfplots}
\usepackage[spanish]{babel}
\usepackage{fancyhdr} 
\usepackage[bottom, norule]{footmisc}
\usepackage{multicol}    
\usepackage{sectsty}    
\usepackage{multicol}
\usepackage{amsmath}
\usepackage{hyperref}
\title{Laboratorio 1}
\date{Octubre 2025}
\pgfplotsset{width=6cm}
\begin{document}
\fancyfoot[C]{}          
\fancyhead{}    
\begin{tikzpicture}[remember picture, overlay]
    \node[anchor=north west, xshift=2.6cm, yshift=-0.5cm] at (current page.north west) 
    {\includegraphics[width=3.5cm]{img/marcaderecha.png}};
    \node[anchor=north east, xshift=-2.6cm, yshift=-1cm] at (current page.north east) 
    {\includegraphics[width=4cm]{img/logo-cfm.png}}; 
\end{tikzpicture}

\begin{center}
\large Laboratorio 1 \\
\vspace{0.3cm}
{\scshape\Huge Tarea 2 \par}
\vspace{0.5cm}
{\scshape\Large Constante elástica de un resorte y Aceleración de gravedad \par}
\vspace{1.5cm}
{\large Autor: Bruno Bustos}
\vspace{0.3cm}
\large 
\\ 
\ttfamily{Universidad de Concepción, Facultad de Ciencias Físicas y Matemáticas.}
\end{center}
\vspace{1cm}

\begin{abstract}

    El objetivo de este informe es determinar la constante elástica de un resorte y la aceleración
    de gravedad mediante la medición experimental de la elongación de un resorte al colgarle diferentes
    masas y dejando caer un carrito a una distancia conocida, peso conocido y midiendo el tiempo que
    tarda en caer, respectivamente.

\end{abstract}

\begin{multicols}{2}

\section{Introducción}


\section{Modelo Teórico}



\section{Materiales}

    \subsection{Materiales para la determinación de la constante elástica de un resorte}

    \subsection{Materiales para la determinación de la aceleración de gravedad}

\section{Procedimiento}

\subsection{Determinación de la constante elástica de un resorte}

\subsection{Determinación de la aceleración de gravedad}

\section{Análisis}

\section{Conclusión}

\section{Referencias}

\end{multicols}
\end{document}