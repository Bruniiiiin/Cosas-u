\documentclass{article}
\usepackage{graphicx} % Required for inserting images
\usepackage{tikz}
\usepackage{circuitikz}
\usepackage{pgfplots}
\usepackage[spanish]{babel}
\usepackage{fancyhdr} 
\usepackage[bottom, norule]{footmisc}
\usepackage{multicol}    
\usepackage{sectsty}    
\usepackage{multicol}
\usepackage{amsmath}
\usepackage{hyperref}
\usepackage{listings}
\usepackage[table,dvipsnames]{xcolor}
\usepackage{listings}
\usepackage{accsupp}
\newcommand{\noncopynumber}[1]{
    \BeginAccSupp{method=escape,ActualText={}}
    #1
    \EndAccSupp{}
}
\lstset{
  language=Python,
  keepspaces=true,
  backgroundcolor=\color{Gray!10},
  basicstyle=\ttfamily\small,
  identifierstyle=\color{Maroon},
  keywordstyle=\color{RoyalBlue},
  commentstyle=\color{OliveGreen!70},
  stringstyle=\color{BrickRed},
  showstringspaces=false,
  % frame=single,
  columns=fullflexible,
  numbers=left,
  numberstyle=\tiny\color{Black}\noncopynumber,
  numbersep=1pt,
  literate=%Algunos caracteres acentuados comunes que dan problemas en listings
         {á}{{\'a}}1 {é}{{\'e}}1 {í}{{\'i}}1 {ó}{{\'o}}1 {ú}{{\'u}}1
         {Á}{{\'A}}1 {É}{{\'E}}1 {Í}{{\'I}}1 {Ó}{{\'O}}1 {Ú}{{\'U}}1
         {à}{{\`a}}1 {è}{{\`e}}1 {ì}{{\`i}}1 {ò}{{\`o}}1 {ù}{{\`u}}1
         {À}{{\`A}}1 {È}{{\`E}}1 {Ì}{{\`I}}1 {Ò}{{\`O}}1 {Ù}{{\`U}}1
         {ñ}{{\~n}}1 {Ñ}{{\~N}}1
}

\lstdefinestyle{terminal}{
  backgroundcolor=\color{black},
  basicstyle=\ttfamily\color{white}\footnotesize,
  frame=single,
  rulecolor=\color{gray!50},
  showstringspaces=false,
  breaklines=true
}

\title{Laboratorio 1}
\date{Octubre 2025}
\pgfplotsset{width=6cm}
\begin{document}
\fancyfoot[C]{}          
\fancyhead{}    
\begin{tikzpicture}[remember picture, overlay]
    \node[anchor=north west, xshift=2.6cm, yshift=-0.5cm] at (current page.north west) 
    {\includegraphics[width=3.5cm]{img/marcaderecha.png}};
    \node[anchor=north east, xshift=-2.6cm, yshift=-1cm] at (current page.north east) 
    {\includegraphics[width=4cm]{img/logo-cfm.png}}; 
\end{tikzpicture}

\begin{center}
\large Laboratorio 1 \\
\vspace{0.3cm}
{\scshape\Huge Tarea 2 \par}
\vspace{0.5cm}
{\scshape\Large Constante elástica de un resorte y Aceleración de gravedad \par}
\vspace{1.5cm}
{\large Autor: Bruno Bustos}
\vspace{0.3cm}
\large 
\\ 
\ttfamily{Universidad de Concepción, Facultad de Ciencias Físicas y Matemáticas.}
\end{center}
\vspace{1cm}

\begin{abstract}

    El objetivo de este informe es determinar la constante elástica de un resorte y la aceleración
    de gravedad mediante la medición experimental de la elongación de un resorte al colgarle diferentes
    masas y dejando caer un carrito a una distancia conocida, peso conocido y midiendo el tiempo que
    tarda en caer, respectivamente.

\end{abstract}

\begin{multicols}{2}

\section{Introducción}

En esta Tarea se aborda la determinación experimental de
dos constantes físicas fundamentales: La aceleración de gravedad $g$
y la constante elástica de un resorte $k$.\\
La aceleración de gravedad es un parámetro fundamental que describe la
fuerza con la que la Tierra atrae a los objetos hacia su centro, mientras
que la constante elástica de un resorte caracteriza la rigidez de un resorte
y su relación  con la fuerza aplicada según la Ley de Hooke.\\
A través de experimentos prácticos, se buscará medir estas constantes de
manera directa, evaluando los posibles errores y comparando los resultados
experimentales con los valores teóricos.

\section{Modelo Teórico}

    \subsection{Constante elástica de un resorte}

        La constante elástica de un resorte $k$ se define como la relación entre la fuerza $F$ aplicada al
        resorte y la elongación $\Delta x$ que experimenta, de acuerdo con la Ley de Hooke \ref{1}:

            \begin{equation}
                F = k \Delta x
            \end{equation}

        Donde:
            \begin{itemize}
                \item $F$ es la fuerza aplicada (en Newtons, N).
                \item $k$ es la constante elástica del resorte (en Newtons por metro, N/m).
                \item $\Delta x$ es la elongación del resorte (en metros, m).
            \end{itemize}
        Tambien tendremos que definir la segunda ley de Newton \ref{1}:

            \begin{equation}
                F = m \cdot a
            \end{equation}

        Donde:
            \begin{itemize}
                \item $F$ es la fuerza neta aplicada al objeto (en Newtons, N).
                \item $m$ es la masa del objeto (en kilogramos, kg).
                \item $a$ es la aceleración del objeto (en metros por segundo al cuadrado, m/s²).
            \end{itemize}

    \subsection{Aceleración de gravedad}

        La aceleración de gravedad $g$ se puede determinar mediante la medición del tiempo $t$ que tarda 
        un objeto en caer, en nuestro caso un carrito con ruedas que desprecian el roce, variando angulos
        de inclinación $\theta$ y conociendo la distancia $x$ recorrida. La relación entre estas variables
        está dada por la siguiente ecuación \ref{1} que sera  :

            \begin{equation}
                x = x_0 + tv_0 + \frac{1}{2} a t^2
            \end{equation}

        Donde:

            \begin{itemize}
                \item $x$ es la posición final (en metros, m).
                \item $x_0$ es la posición inicial (en metros, m).
                \item $v_0$ es la velocidad inicial (en metros por segundo, m/s).
                \item $a$ es la aceleración del objeto (en metros por segundo al cuadrado, m/s²).
                \item $t$ es el tiempo de caída (en segundos, s).
            \end{itemize}

        En el caso de un plano inclinado, la aceleración $a$ está relacionada con la aceleración de 
        gravedad $g$ y el ángulo de inclinación $\theta$ mediante la siguiente relación \ref{1}:

            \begin{equation}
                a = g \sin(\theta)
            \end{equation}
        
\section{Materiales}

    \subsection{Materiales para la determinación de la constante elástica de un resorte}
    \begin{itemize}
        \item Resorte
        \item Soporte para resorte
        \item Pesas de masa conocida
        \item Regla
    \end{itemize}
    \subsection{Materiales para la determinación de la aceleración de gravedad}
    \begin{itemize}
        \item Carrito con ruedas
        \item Rampa inclinada
        \item Regla
        \item Cronómetro
    \end{itemize}

\section{Procedimiento}

\subsection{Determinación de la constante elástica de un resorte}

    Primero se cuelga el resorte del soporte y se mide su longitud inicial sin ninguna masa colgada.
    Luego, se cuelgan diferentes masas conocidas al resorte y se mide la elongación del resorte 
    para cada masa. Se repite este proceso para 14 masas diferentes, registrando 14 elongaciones distintas.
    \\ Tendremos una tabla con las masas y las elongaciones correspondientes:
    \begin{center}
        \begin{tabular}{|c|c|}
            \hline
            Masa (kg) & Elongación (m) \\
            \hline
            25.8 & 0.3 \\
            40.6 & 0.4 \\
            47.7 & 0.9 \\
            90.0 & 2.3 \\
            105.4 & 3.0 \\
            187.2 & 5.3 \\
            250.0 & 7.5 \\
            339.0 & 10.2 \\
            416.8 & 12.8 \\
            499.7 & 15.4 \\
            749.5 & 23.6 \\
            797.5 & 24.9 \\
            999.6 & 31.1 \\
            1122.8 & 35.5 \\
            \hline
        \end{tabular}
    \end{center}
\subsection{Determinación de la aceleración de gravedad}

    Se coloca la rampa inclinada en un ángulo conocido $\theta$. Se mide una distancia $d$ desde
    donde comienza el carrito hasta el punto donde se detiene. Se suelta el carrito y se registra
    el tiempo $t$ que tarda en recorrer la distancia $d$. Este proceso se repite para diferentes
    ángulos de inclinación. Se varía el ángulo 8 veces y para cada ángulo se registran 10 tiempos,
    se tomara el promedio de los tiempos para cada ángulo.
    \\ Tendremos una tabla con los ángulos y los tiempos promedios correspondientes:

    \begin{center}
        \begin{tabular}{|c|c|}
            \hline
            Ángulo (Rad) & Promedio Tiempo (s) \\
            \hline
            1 & 2.89 \\
            4 & 1.48 \\
            6 & 1.15 \\
            8 & 0.78 \\
            10 & 0.84 \\
            12 & 0.81 \\
            15 & 0.62 \\
            18 & 0.63 \\
            \hline
        \end{tabular}
    \end{center}

\section{Análisis}

    \subsection{Determinación de la constante elástica de un resorte}



    \subsection{Determinación de la aceleración de gravedad}

    

\section{Conclusión}

\section{Referencias}
    \begin{enumerate}
    
        \item \label{1} Resnick, R., Halliday, D., \& Krane, K. (1988). \textit{Physics, Vol. 1} (4th ed.). John Wiley \& Sons. 
    
    \end{enumerate}
\end{multicols}
\end{document}