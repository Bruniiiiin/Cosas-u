\documentclass{article}
\usepackage{graphicx} % Required for inserting images
\usepackage{tikz}
\usepackage{circuitikz}
\usepackage{pgfplots}
\usepackage[spanish]{babel}
\usepackage{fancyhdr} 
\usepackage[bottom, norule]{footmisc}
\usepackage{multicol}    
\usepackage{sectsty}    
\usepackage{multicol}
\usepackage{amsmath}
\usepackage{hyperref}
\title{Laboratorio 1}
\date{Octubre 2025}
\pgfplotsset{width=6cm}
\begin{document}
\fancyfoot[C]{}          
\fancyhead{}    
\begin{tikzpicture}[remember picture, overlay]
    \node[anchor=north west, xshift=2.6cm, yshift=-0.5cm] at (current page.north west) 
    {\includegraphics[width=3.5cm]{img/marcaderecha.png}};
    \node[anchor=north east, xshift=-2.6cm, yshift=-1cm] at (current page.north east) 
    {\includegraphics[width=4cm]{img/logo-cfm.png}}; 
\end{tikzpicture}

\begin{center}
\large Laboratorio 1 \\
\vspace{0.3cm}
{\scshape\Huge hola \par}
\vspace{0.5cm}
{\scshape\Large Constante de elasticidad y Aceleración de gravedad \par}
\vspace{1.5cm}
{\large Autor: Bruno Bustos}
\vspace{0.3cm}
\large 
\\ 
\ttfamily{Universidad de Concepción, Facultad de Ciencias Físicas y Matemáticas.}
\end{center}
\vspace{1cm}

\begin{abstract}
En este informe se analiza la relación entre la diferencia de potencial eléctrico y la intensidad de corriente en tres circuitos distintos con estructuras en serie y paralelo, con el propósito de verificar experimentalmente la Ley de Ohm. Se realizaron mediciones de voltaje y corriente sobre resistencias y una ampolleta, registrando los datos. Los resultados obtenidos evidenciaron un comportamiento lineal en las resistencias, confirmando su naturaleza óhmica, mientras que la ampolleta presentó una respuesta no lineal, identificándose como un conductor no óhmico.
Con esto se corroboraron las expresiones teóricas que describen las resistencias equivalentes en cada configuración, y se comprobó experimentalmente la efectividad como modelo de la Ley de Ohm. 


\end{abstract}

\begin{multicols}{2}

\section{Introducción}

\section{Marco Teórico}

\section{Materiales}

\section{Montaje}

\section{Procedimiento}

\subsection{Circuito Resistivo Simple}

\subsection{Circuitos con combinación de resistencias}

\section{Análisis}

\section{Conclusión}

\section{Referencias}

\end{multicols}
\end{document}